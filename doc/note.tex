background subtraction based on CNN

第一步测试多个数据集融合训练

createImdb做好了,现在测试单个视频训练

单个视频训练都是可以用于测试。但是只采前80帧效果不好

测试融合之后训练的的网络

现在的数据是 81 × 255 × 3 = 61965
对应的标签目前是 4个
所以可能性总数是 81 × 255 × 3 × 4 = 247860.
即便是没有取绝对值255 × 2 = 510
总可能是 81 × 255 × 2 × 3 × 4 = 495720.
但现在输入的训练是 664300 > 495720

对应的标签目前是 4个
所以可能性总数是 81 × 255 × 3 × 4 = 247860.
即便是没有取绝对值255 × 2 = 510
总可能是 81 × 255 × 2 × 3 × 4 = 495720.
但现在输入的训练是 664300 > 495720
所以数据肯定有重复的。 
664300 : 495720

回看mnist的数据。10 28 × 28 数据范围-255~255
28 × 28 × 10 × 510 = 
3998400 : 70000

imdb = load('./data/imdb_9_3_1410_highway_test.mat');
   0.8016    0.9995    0.8897

imdb = load('./data/imdb_9_3_910_pedestrians_test.mat');
    0.9610    0.6987    0.8091

imdb = load('./data/imdb_9_3_910_office_test.mat');
    0.9487    0.9939    0.9708

imdb = load('./data/imdb_9_3_910_pets2006_test.mat');
    0.9772    0.9460    0.9614

输入数据远远大于数据类型的种类。
放起了,数据输入大小只能是9×9,硬件限制。

即便如此,均值也到了0.9078

测试一下另一个数据集


imdb = load('./data/imdb_9_3_910_Board_test.mat');
    0.1029    0.9550    0.1857

imdb = load('./data/imdb_9_3_110_CAUIAR1_test.mat');
    0.6647    0.7061    0.6848

imdb = load('./data/imdb_9_3_110_CAUIAR2_test.mat');
    0.9636    0.2015    0.3333

均值为 0.4013

视频的主要颜色不一样,种类不一样,物体的运动速度不一样,所以直接迁移不行。

数据重复量也不是特别大。写测试程序

一个网络不足3M,59 × 3 = 177 M 硬件足够!

分开训练


联合训练的结果
pets2006
    0.9077    0.8387    0.8718

highway
    0.6158    0.9967    0.7612

office
    0.8356    0.9820    0.9029

pedestrians
    0.9587    0.6122    0.7472

baseline的均值都是0.92

单独训练
算法的定位。需要的训练少。单独分开训练,不然谁都搞不过。
网络一个才2M,higwahy的数据也才20M,限制图像大小

单独训练的结果是
highway
    0.8520    0.9595    0.9026
测试帧中完全不含训练帧

扩大数据,增加到18
    0.8708    0.9787    0.9216
    有明显的提升,而且才训练了5epoch。

训练到第20个epoch的时候,到了0.94
    0.9003    0.9850    0.9407

但是到第25个epoch时,值又回到了
    0.8722    0.9917    0.9282

硬件限制,不能把网络弄得太好,
测试15×15i
第8个epoch的时候到了
    0.9031    0.9887    0.9440
第20个ephco的时候
    0.9053    0.9920    0.9467

所有视频的结果是  highway
    0.8833    0.9926    0.9348

虽然知道不行,但还是有点不信邪。试下其他的视频
office 的结果i
    0.5922    0.5984    0.5953
    不加入训练,不能迁移

测试12×12的,如果能上92,就用12
效果是:
    0.7984    0.9834    0.8813

    更差了。测试9×9

本质上,领域采样,只是为了提高时序的不足。

用回9×9 81帧 0 neighborhoods
    0.9154    0.9828    0.9479
比之前的还高 
所有视频的结果是:
    0.9102    0.9819    0.9447

直接测试dynamic background

highway 2帧间隔的效果
    0.9137    0.9776    0.9446

fountain02
    0.7820    0.9548    0.8598


nightvideos 虽然差,也有0.5左右
    0.7084    0.4964    0.5838

目标 日死那篇2017的pami 让你不回我邮件!!!

Approach	


Baseline	            0.9534	    0.94
Dynamic Background	    0.9120	    0.88
Camera Jitter	        0.8503	    0.66
Intermittent Motion	    0.8349	    0.60
Shadow	                0.8930	    0.81
Thermal	                0.8579	    0.76
Avg                     0.8836      0.76

pami只有0.784的文章
0.784
http://www.vis.uni-stuttgart.de/forschung/informationsvisualisierung-und-visual-analytics/visuelle-analyse-videostroeme/sabs.html

dynamicBackground
baseline
nightVideos 0.58

明天测试nightVideos 3×3

boulevard
增加背景帧之后,依然是
    0.6408    0.5367    0.5842

    测试增加领域信息
    0.5792    0.5998    0.5893
将数据帧增加到 100
    0.7966    0.5757    0.6684
100 10epoch
    0.5011    0.8569    0.6324

100 5epoch
    0.7827    0.5097    0.6173


没增加多少,放弃
    不能这么被卡着,继续下一个,

shaodow
    0.9435    0.7221    0.8181

Threaml 5 epoch
    0.6501    0.9275    0.7644
    10 epoch
    0.7072    0.8603    0.7763
    20 epoch
    0.7141    0.8305    0.7679

Intermittent Motion:
    0.3739    0.8445    0.5183


Baseline	            0.9534	    0.94
Dynamic Background	    0.9120	    0.88
Camera Jitter	        0.8503	    0.66
Intermittent Motion	    0.8349	    0.60
Shadow	                0.8930	    0.81
Thermal	                0.8579	    0.76
Avg                     0.8836      0.78

优化下代码


将数据帧增加到 100
    0.7966    0.5757    0.6684
100 10epoch
    0.5011    0.8569    0.6324

100 5epoch
    0.7827    0.5097    0.6173

225 3 epoch 
    0.7564    0.5632    0.6457

255 10 epoch 200 
    0.8229    0.6083    0.6995

255 10 epch 50 间隔
    0.8469    0.5502    0.6670

255 20 epoch 100 间隔
    0.7770    0.6519    0.7090

225 20 epoch 50 间隔
    0.8034    0.6046    0.6899
